\documentclass[]{article}
\usepackage{lmodern}
\usepackage{amssymb,amsmath}
\usepackage{listings}
\usepackage{graphicx}
\usepackage{subcaption}
\usepackage{hyperref}

\begin{document}
\section{Convolutional Networks}
\begin{center}
  \emph{By: Jussi Martin}
\end{center}
\subsection{Preface}
This post is written for the AI Helsinki study group \emph{Image and Video Statistics}.
It is based on the Chapter 9 of the book \emph{Deep Learning} by Ian Goodfellow,
Yoshua Bengio and Aaron Courville, which is under preparation and available
online at \url{http://www.deeplearningbook.org}.

We will assume the reader to be familiar with basic components of artificial
neural networks, such as weights and biases. Some mathematical preliminaries are
introduced in the text.

Our goal is to introduce basic concepts of the convolutional neural networks,
explain where they are used and what kind of benefits the convolutional
structure offers.

% Skeleton for the post: %

\subsection{Definition}
We begin by introducing the mathematical notion of convolution. Let $x(t)$ and $w(t)$
two functions then we define their convolution $x * w$ as
\[
(x * w)(t) = \int x(a)w(t - a)da.
\]
In practical applications $x$ could be some signal depending on the time $t$ and
$w$ a filter applied to it.

Obiously the data for which the convolution is applied usually is obtained only
as a discrete input, in these cases we define the convolution by using
summation:
\[
(x * w)(t) = \sum_{a = -\infty}^{\infty} x(a)w(t - a).
\]
Moreover, the data might be two dimensional, which is the case with images, then
we use the follwing definition:
\[
S(i, j) = (I * K)(i, j) = \sum_m \sum_n I(m , n) K(i - m, j - n)
\]
where $I$ and $K$ represent two images and $i$ and $j$ are the pixel coordinates.

In practice the input grid is finite, the filter is defined on a smaller grid and
usually only applied when its coordinates are fully in contained in the input grid
These technical details will be covered soon.
% One special property of convolution is that it is commutative, that is
% \[
%  (K * I) = (K * I).
% \]
% Which is relatively easy to derive, especially for the discrete cases.
% When defined without kernel-flipping:
% \[
% S(i, j) = \sum_m \sum_n I(i + m , j + n) K(m, n)
% \]
% (also known as cross-correlation).

We say that a neural network is convolutional if it uses convolutional filtering
in at least one of its layers.

\subsection{Benefits of Convolutional Networks}
Convolutional networks are particularlly useful in speech recognition and image
classification, since they can learn to be unsensitive to local translations in the
input. For example, convolutional network can learn to classify different faces as
been an image of a face even though individual faces have slight variation in
the positions of different facial features.

Besides the classificational aspescts, convolution is also effective in the sence
that it uses fewer connections and shared parameters, which reduces the
computational cost and memory requirements. In deep networks this difference is
significant.

We will examine sparse connectivity and parameter sharing later, after we
have first introduced tensors and shown how differnt type of convolutions are
represented with them.

\subsection{Tensors}
We adopt the Deep Learning book's convention of defining tensors as
multidimensional arrays of real numbers.

Namely, 0-D tensors are just real numbers, 1-D tensors are arrays
\[
T = (T_1, \ldots, T_n)
\]
of real numbers, 2-D tensors are arrays
\[
T = ((T_{1,1}, \ldots, T_{1,n}), \ldots (T_{m,1}, \ldots, T_{m,n}))
\]
of arrays of real numbers (with mutually equal lenght), and so on.

Basically our tensors can be viewed as D-dimensional grids of real numbers.
Example in 2-D:
\[
((1, 2, 3), (4, 5, 6), (7, 8, 9)) \quad \simeq \quad
\begin{array}{c|c|c}
  1 & 2 & 3 \\
  \hline
  4 & 5 & 6 \\
  \hline
  7 & 8 & 9
 \end{array}
\]
In general case the sizes of the axes do not need to match.

% There are also more elaborate ways of defining tensors, that take into account
% the type of the tensor, but these are not needed for our purposes.

\subsection{Convolution via Tensors}
*Stride\\
*Channels\\
*Formulas, etc.\\
(convolution with $1\times 1$)... traditional fully connected layer can be seen
as other extreme case, with zero stride and kernel grid size been equal to the
input grid size.

\subsection{Examples (submerge with previous section?)}
(Pic of 2-D convolution, should be earlier?)

\subsection{Sparse Connectivity and Parameter Sharing}
As mentioned earlier, in a convolutional layer there are fewer connections
between the nodes compared to fully connected layers and the same weights are used
with multiple connections. We will now illuminate this with some visualizations
and estimates.

Namely, only as many input nodes as there are weights in the convolution kernel
are connected to every single node in the output nodes and vice versa.

\[
\text{Pic: 1-D, connections from inputs to single node in the outputs}
\]

\[
\text{Pic: 1-D, connections from single input to nodes in the outputs}
\]

This is in drastic contrast to fully connected layers where there are connections
from every input node to a given output node and vice versa.

For example, if we have an image with $28 \times 28$ pixels grid, like the images
in the MNIST data set have, and we convolve it with $3 \times 3$ filter. Then
there are only $3 \cdot 3 = 9$ connections from input nodes to single output node,
where as in fully connected layer there would be $28 \cdot 28 = 784$ connections
to single output node. And similarly with roles of the input and output nodes reversed.

Here is an illustration of how the weights of the convolution kernel are shared
between connections:

\[
\text{Pic: illustration of the parameter sharing}
\]

Sometimes even the biases are shared in similar manner, but this type of
parameter sharing is not often used. Also here the difference to fully connected
layer is significant. If fully connected layer has $n$ input nodes and $m$ output
nodes, then there are $n \times m$ weight parameters in the layer, where as
convolutional layer has only as meny weight parameters as the convolution kernel
has.

\subsection{Zero Pading}
One drawback of using convolutional layers is that the layers shrink, which
opposes limit to the depth of the network.
\[
\text{Pic: 1-D network with shrinking layers}
\]
This however can be compansated by
adding zeros to the boundary of each neuron layer. That way we can prevent the
sizes of the layers shrinking and obtain deeper networks.
\[
\text{Pic: 1-D network with zero pading and same size layers}
\]
The method discribed above is known as zero padding. One way to do it is to add
just enough zeros that the layer sizes stays the same. Other more extreme option
is to add so many zeros that the convolution can visit every neuron $k$ times,
where $k$ is the size of the convolution kernel.
\[
\text{Pic: 2-D, original layer and layer with minimal pading}
\]
\[
\text{Pic: 2-D, original layer and layer with maximal pading}
\]
One may notice that zero pading makes the outputs near the boundary to have less
information from the inputs, especially if there is maximal amount of pading
used. This may be compensated to some extend if the biases are not shared (which
usually is the case).

\subsection{(Max) Pooling}
Typically pooling is used for down sampling, which means that the grid size of
the data is reduced.
\[
\begin{array}{c|c|c|c}
  0 & 1 & 3 & 9\\
  \hline
  4 & 8 & 6 & 2\\
  \hline
  7 & 4 & 2 & 0\\
  \hline
  4 & 3 & 5 & 1\\
 \end{array}
 \quad \mapsto \quad
 \begin{array}{c|c}
   8 & 9\\
   \hline
   7 & 5
  \end{array}
 \]
\begin{center}Example in 2-D: max pooling with $2\times2$ kernel and stride = 2.\end{center}
Another benefit of max pooling with down sampling is that it is not so sensitive
to small translations. If we take to grids, which are small translations of each
other, and compare the pooled grid, we see that many of the values are the same.
\[
\text{Pic: Without Shift, Shifted}
\]

\subsection{Growing Receptive Fields (optional ?)}
For a given node the nodes in the previous layers which are directly connected
to it are known as receptive field.

Although the receptive field for single convolutional layer is small due to sparse
connections, it grows when there are more layers in the network and it can even
cover the whole input grid if the network has enough layers.

\subsection{Convolutional Network Components}
Typically: (convolution, Relu, pooling) x $N_1$, followed by FC x $N_2$

% \subsection{Max Pooling}
% Pictures: Without Shift, Shifted

\subsection{Example of Learned Invariances}
Explanation (no pics)
% \subsection{Pooling with Down Sampling}

% \subsection{Examples of Architectures}
% We give here two examples of convolutional network architectures. Namely, one
% which is fairly classical and another which is from this year (2016).
%
% ...Explanations and pics...

\subsection{Further Topics (Optional ?)}
In this last section we will mention briefly some of the topics presented in the
Chapter 9 of the Deep Learning book that we have not yet covered.

There are variants of the usual way of doing the convolution...
\newline
*Comparison of Local Connections, Convolution, and FC (Optional)\\
*Tiled Convolution (optional)\\
*Recurent Convolutional Networks

%\subsection{Convolution with Strides}

%\subsection{Zero Pading Enables Deeper Networks}

% \subsection{Comparison of Local Connections, Convolution, and Full Connections}
% Pictures: Local, Convolution, FC

% \subsection{Partial Connectivity Between Channels}

% \subsection{Tiled Convolution}
% Pictures: Local Connection, Tiled Convolution, Traditional Convolution

% \subsection{Recurent Convolutional Networks}

%\subsection{Gabor Functions (optional)}

%\subsection{Gabor-like Learned Kernels (optional)}

\subsection{Links}
* Our main source:

\url{http://www.deeplearningbook.org}\\
* Video with very basics of convolutional networks:

\url{https://www.youtube.com/watch?v=JiN9p5vWHDY}\\
* Video with slightly more advanced level introduction:

\url{https://www.youtube.com/watch?v=FmpDIaiMIeA}\\
* Video of a conference presentation:

\url{https://www.youtube.com/watch?v=AQirPKrAyDg}



\end{document}
