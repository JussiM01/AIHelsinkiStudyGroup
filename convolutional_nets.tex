\documentclass[]{article}
\usepackage{lmodern}
\usepackage{amssymb,amsmath}
\usepackage{listings}
\usepackage{graphicx}
\usepackage{subcaption}
\usepackage{hyperref}

\begin{document}
\section{Convolutional Networks}
This is a post written for the AI Helsinki study group \emph{Image and Video Statistics}.
It is based on the Chapter 9 of the book \emph{Deep Learning}.

% Skeleton for the post: %

\subsection{Definition}

\subsection{Use of Convolutional Networks}

\subsection{Benefits}

\subsection{Examples}

\subsection{Sparse Connectivity}

\subsection{Growing Receptive Fields}

\subsection{Parameter Sharing}

\subsection{Convolutional Network Components}

\subsection{Max Pooling}
Pictures: Without Shift, Shifted

\subsection{Example of Learned Invariances}

\subsection{Pooling with Down Sampling}

\subsection{Examples of Architectures}

\subsection{Convolution with Strides}

\subsection{Zero Pading Enables Deeper Networks}

\subsection{Comparison of Local Connections, Convolution, and Full Connections}
Pictures: Local, Convolution, FC

\subsection{Partial Connectivity Between Channels}

\subsection{Tiled Convolution}
Pictures: Local Connection, Tiled Convolution, Traditional Convolution

\subsection{Recurent Convolutional Network}

\subsection{Gabor Functions (optional)}

\subsection{Gabor-like Learned Kernels (optional)}

\section{Tensors}

We adopt the Deep Learning book's convention of calling multidimensional arrays
of real numbers as tensors.

Namely, 0-D tensors are just real numbers, 1-D tensors are arrays
\[
T = (T_1, \ldots, T_n)
\]
of real numbers, 2-D tensors are arrays
\[
T = ((T_{1,1}, \ldots, T_{1,n}), \ldots (T_{m,1}, \ldots, T_{m,n}))
\]
of arrays of real numbers (with mutually equal lenght), and so on.

Basically our tensors can be wieved as D-dimensional grids of real numbers.
Example in 2-D:
\[
((1, 2, 3), (4, 5, 6), (7, 8, 9)) \quad \simeq \quad
\begin{array}{c|c|c}
  1 & 2 & 3 \\
  \hline
  4 & 5 & 6 \\
  \hline
  7 & 8 & 9
 \end{array}
\]

There are also more elaborate ways of defining tensors, that take into account
the type of the tensor, but these are not needed for our purposes.

\section{References/Links}

\end{document}
