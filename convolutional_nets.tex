\documentclass[]{article}
\usepackage{lmodern}
\usepackage{amssymb,amsmath}
\usepackage{listings}
\usepackage{graphicx}
\usepackage{subcaption}
\usepackage{hyperref}

\begin{document}
\section{Convolutional Networks}
\begin{center}
  \emph{By: Jussi Martin}
\end{center}
\subsection{Preface}
This blog post is written for the \emph{Image and Video Statistics} study group
held by the AIHelsinki Academia. It is based on the Chapter 9 of the book
\emph{Deep Learning} by Ian Goodfellow, Yoshua Bengio and Aaron Courville, which
is available online at \url{http://www.deeplearningbook.org}.

We will assume the reader to be familiar with basic components of artificial
neural networks, such as weights and biases. Some mathematical preliminaries are
introduced in the text. Our goal is to introduce basics of convolutional neural
networks and explain some of their properties.

Convolutional networks are particularly useful in image classification, since
they can learn to be invariant to local translations in the input. For example,
convolutional network can learn to classify image of a face correctly despite
individual faces having slight variation in the positions of facial features.

Besides the classificational aspects, convolution is also effective in the sense
that it uses fewer connections and shared parameters, which reduces the
computational cost and memory requirements and improves the statistical efficiency.
In deep networks this difference is significant.

\subsection{Definition}
We begin by recalling the mathematical notion of convolution.
Since we are dealing with images, the input data is two dimensional and the convolution
$(I * K)$ is defined as
\[
(I * K)(i, j) = \sum_m \sum_n I(m , n) K(i - m, j - n)
\]
where $I$ is an images, $K$ is a filter applied to it, $i$ and $j$ are the
pixel coordinates. One special property of convolution is that it is commutative, that is
\[
 (K * I) = (K * I).
\]
Which is relatively easy to derive.

In practice the input grid is finite, the filter is defined on a smaller grid and
usually only applied when its coordinates are fully contained in the input grid.

Many machine learning libraries actually defined convolution as
\[
(I * K)(i, j) = \sum_m \sum_n I(i + m , j + n) K(m, n),
\]
which is also known as cross-correlation or convolution without kernel-flipping.
It is not commutative, but from a practical point of view this dose not make any difference
since commutativity is not needed and the network can learn its parameters with
respect to either of these convolutions.

We say that a neural network is convolutional if it uses convolution in place of
general matrix multiplication in at least one of its layers.

\subsection{Convolutional Networks Structure}
Here we give an example layout for convolutional networks structure. We will also
consider what type of outputs convolutional network can have. Individual layers are
presented later.

Typically the first layer in a convolutional network is convolutional. After this
the output is passed trough a nonlinearity, at this point there might be a pooling
layer. This type of structure can then be repeated multiple times. The final layers
are usually fully connected.

Here is an example network, which was one of the networks used by the VGG team
in the ImageNet 2014 Challenge:

\begin{center}
\begin{tabular}{|c|}
  \hline
  input, 224 $\times$ 224 RGB image\\
  \hline
  convolutional, kernel width 3, channels 64\\
  \hline
  convolutional, kernel width 3, channels 64\\
  \hline
  max pooling\\
  \hline
  convolutional, kernel width 3, channels 128\\
  \hline
  convolutional, kernel width 3, channels 128\\
  \hline
  max pooling\\
  \hline
  convolutional, kernel width 3, channels 256\\
  \hline
  convolutional, kernel width 3, channels 256\\
  \hline
  convolutional, kernel width 3, channels 256\\
  \hline
  convolutional, kernel width 3, channels 256\\
  \hline
  max pooling\\
  \hline
  convolutional, kernel width 3, channels 512\\
  \hline
  convolutional, kernel width 3, channels 512\\
  \hline
  convolutional, kernel width 3, channels 512\\
  \hline
  convolutional, kernel width 3, channels 512\\
  \hline
  max pooling\\
  \hline
  convolutional, kernel width 3, channels 512\\
  \hline
  convolutional, kernel width 3, channels 512\\
  \hline
  convolutional, kernel width 3, channels 512\\
  \hline
  convolutional, kernel width 3, channels 512\\
  \hline
  max pooling\\
  \hline
  fully connected, nodes 4096\\
  \hline
  fully connected, nodes 4096\\
  \hline
  fully connected, nodes 1000\\
  \hline
  soft-max\\
  \hline
\end{tabular}
\end{center}
All the hidden layers are followed by ReLu (defined later),
which is not shown for the sake of brevity. For details, see
\url{https://arxiv.org/pdf/1409.1556.pdf}.

\subsection{Tensors and Convolutional Layer}
We adopt the Deep Learning book's convention of defining tensors as
multidimensional arrays of real numbers.

Namely, 0-D tensors are just real numbers, 1-D tensors are arrays
\[
T = (T_1, \ldots, T_n)
\]
of real numbers, 2-D tensors are arrays
\[
T = ((T_{1,1}, \ldots, T_{1,n}), \ldots (T_{m,1}, \ldots, T_{m,n}))
\]
of arrays of real numbers (with mutually equal length), and so on.

Basically our tensors can be viewed as D-dimensional grids of real numbers.
Example in 2-D:
\[
((1, 2, 3), (4, 5, 6), (7, 8, 9)) \quad \simeq \quad
\begin{array}{|c|c|c|}
  \hline
  1 & 2 & 3 \\
  \hline
  4 & 5 & 6 \\
  \hline
  7 & 8 & 9 \\
  \hline
\end{array}
\]
In general case the sizes of the axes do not need to match.

Typically convolutional network has several channels which may correspond to different
colors in the image or they may originate from same input image which is processed
in different ways in parallel.

We now assume for the sake of simplicity that the input grid has spatial shape of
$a \times a$ grid and the kernel grid has spatial shape of $b \times b$ grid, where
$a$ and $b$ are independent of the channel index. We also denote by $c$ and $d$
the number of input and output channels in the layer.

Output tensor $Z$ of the convolutional layer at pixel $(j, k)$ of the channel $i$
is now defined by
\[
Z_{i, j, k} = \sum_{l=1}^c \sum_{m = 1}^b \sum_{n=1}^b
V_{l, j + m -1, k + n -1} K_{i, l, m, n}
\]
where $V$ is the input tensor with $l$ as its channel index, $K$ is the weight tensor with
$K_{i, l, m, n}$ representing the weight of the connection from node in output
channel $i$ to node in input channel $l$, with an offset of $m$ rows and $n$ columns
between the channels.

Also, the output node $Z_{i, j, k}$ is only defined when
\[
1 \le j + m - 1 \le a \quad \text{and} \quad 1 \le k + n - 1 \le a
\]
for all terms in the summation. This is due to the requirement that the kernel fits
in to the input grid, which now looks different since we are using convolution
without kernel-flipping. Since $m$ and $n$ are at most $b$, we see that
\[
1 = \max\{2 - b, 1\} \le j \le a - b + 1\quad \text{and the same way} \quad 1 \le k \le a - b + 1.
\]
Namely, this means that the convolution has shrunk the spatial width of the data grid
from $a$ to $a - b + 1$. This is a problem since it opposes a limit to the depth of
the network. It can however be compensated by adding artificially zeros to the
boundary of the input layer. This way it is possible to obtain deeper networks. The
method is known as \emph{zero padding}.

The convolution can also be chosen to accept input from a periodical sub-grid.
This is known as convolution with a stride $s$ and defined by
\[
Z_{i, j, k} = \sum_{l=1}^c \sum_{m = 1}^b \sum_{n=1}^b
V_{l, (j -1) s + m, (k-1) s + n} K_{i, l, m, n}
\]
with the same type of validity conditions required as before. Similar stride is used
also for pooling which we define later.

\subsection{Nonlinearities}
Some of the typical nonlinearities used in convolutional networks are:

Rectified linear unit also known as \emph{ReLu}, defined by
\[
f(x) = \max(x, 0).
\]

Logistic funtion also known as \emph{sigmoid}, defined by
\[
f(x) = \frac{1}{1 + \exp(-x)}.
\]

Hyperbolic tangent function also known as \emph{tanh}, defined by
\[
f(x) = \frac{\exp(x) - \exp(-x)}{\exp(x) + \exp(-x)}.
\]

A nonlinearity is applied after convolutional layer. Its output tensor $T$ is defined by
\[
T_{i,j,k} = f(Z_{i, j, k} + b_{i,j,k})
\]
where $Z$ is the output tensor of the convolutional layer and $b_{i, j, k}$ is a
bias unit, which is often shared between units in the same channel, that is
\[
b_{i, j, k} = b_i \quad \text{for all $j$ and $k$}
\]
where $j$ and $k$ are the pixel indices and $i$ is the channel index.

In some case, for example, if objects in a set of images are known to be in the
center of the images, it may be more convenient to allow the biases to vary with
respect to location.

\subsection{Pooling}
Typically pooling is used for down sampling, which means that the grid size of
the data is reduced. Method that is most often used nowadays is max pooling and
its output tensor $T$ is defined by
\[
T_{i, j,k} = \max \{V_{i, s j + \alpha, s k + \beta} |
\text{ for $(\alpha, \beta) \in \{ 1,\ldots, b\} \times  \{ 1,\ldots, b\} $} \}
\]
where $s$ is the stride, $b$ is the spatial width of the max pooling kernel. Often
max pooling is used with $s = b > 1$, which has the effect of shrinking the data
grid by factor $s$.

Example in 2-D, max pooling with both $s$ and kernel width set to 2:
\[
\begin{array}{|c|c|c|c|}
  \hline
  0 & 1 & 3 & 9\\
  \hline
  4 & 8 & 6 & 2\\
  \hline
  7 & 4 & 2 & 0\\
  \hline
  4 & 3 & 5 & 1\\
  \hline
 \end{array}
 \quad \stackrel{\text{Pooling}}{\longrightarrow} \quad
 \begin{array}{|c|c|}
   \hline
   8 & 9\\
   \hline
   7 & 5\\
   \hline
  \end{array}
 \]

\subsection{Sparse Connections}
One may notice that an output node in a convolutional or a pooling layer is only
connected to a set of nodes of the size of the convolution or pooling kernel.
Similarly, the input nodes are connected to only as many output nodes as is the size
of the kernel (or some times even less if a stride is used). This is in great contrast
to fully connected layers where every node in the input is connected to every node
in the output nodes.


\subsection{Receptive Field}
For a given node the nodes in the previous layers which are directly connected
to it are known as receptive field.

Although the receptive field for a node in the early layers of the network is small
due to sparse connections, it is larger for nodes in the deeper layers of the network
and it can even cover the whole input image if the network has enough layers.

\end{document}
