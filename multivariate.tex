\documentclass{beamer}
%
% Choose how your presentation looks.
%
% For more themes, color themes and font themes, see:
% http://deic.uab.es/~iblanes/beamer_gallery/index_by_theme.html
%
\mode<presentation>
{
  \usetheme{default}      % or try Darmstadt, Madrid, Warsaw, ...
  \usecolortheme{default} % or try albatross, beaver, crane, ...
  \usefonttheme{default}  % or try serif, structurebold, ...
  \setbeamertemplate{navigation symbols}{}
  \setbeamertemplate{caption}[numbered]
}

\usepackage[english]{babel}
\usepackage[utf8x]{inputenc}

\title[Multivariate Probability and Statistics]{Multivariate Probability and Statistics}
\author{Jussi Martin}
%\institute{Where You're From}
\date{October 24, 2016}

\begin{document}

\begin{frame}
  \titlepage
\end{frame}

% Uncomment these lines for an automatically generated outline.
%\begin{frame}{Outline}
%  \tableofcontents
%\end{frame}

% \section{Introduction}
%
% \begin{frame}{Introduction}
%
% \begin{itemize}
%   \item Your introduction goes here!
%   \item Use \texttt{itemize} to organize your main points.
% \end{itemize}
%
% \vskip 1cm
%
% \begin{block}{Examples}
% Some examples of commonly used commands and features are included, to help you get started.
% \end{block}
%
% \end{frame}

\section{Introduction}

\begin{frame}{Introduction}
  This presentation is a brief summary of some of the topics introduced in
  chapter 4 of the book \emph{Natural Image Statistics} (reference at
  the last slide).

  Since there won't be enough time to cover the whole chapter 4, I decide to go
  trough the things that are needed for understanding the \emph{Bayes' rule},
  which I think is one of the core contents in this chapter.

  I will also introduce \emph{expectation}, \emph{variance}, \emph{covariance},
  \emph{likelihood}, \emph{log-likelihood}, \emph{maximum likelihood estimate}
  and \emph{maximum a posteriori estimates} in these slides, since some of them
  are needed in the exercises.
\end{frame}

\section{Random Variables}

\begin{frame}{Random Variables}
 Discrete random variable is a variable which can have any value from some given
 countable set with some given probabilty. Example: outcome of throwing a dice.
 The outcome can have any integer value from 1 to 6, each with the probability
 $\frac{1}{6}$.

 Continuos random variable is a variable which can have any value from some given
 continuous range of values, with some given probabilities for the value been in
 any given interval. Example: uniform probability distribution on the interval
 $[0, 1]$. Now the outcome is in an interval $[a, b]$ with the probability $b-a$
 for any given values $0 \le a \le b \le 1$.
\end{frame}

\section{Random Vectors}

\begin{frame}{Random Vectors}
  Random vector is a $n$-tuple of random variables. Each of these can be either
  similar or different types of variables. Formally:
  \[ \mathbf{z} =  \begin{pmatrix} z_1 \\ z_2 \\ \vdots \\ z_n \end{pmatrix} \]
  where $\mathbf{z}$ is the random vector and the random variables $z_i$ are
  its components.
\end{frame}

\section{Probability Density Function (pdf)}

\begin{frame}{Probability Density Function (pdf)}
  Let $P(\text{$z$ is in [a, b]})$ be the probability that the value of a
  random variable $z$ is in the interval $[a, b]$, then the probabilty density
  function $p_z(a)$ can be defined approximately as follows:
  \[ p_z(a) \approx \frac{P(\textnormal{$z$ is in $[a, a + \nu]$})}{\nu}\]
  for very small values of $\nu$.

  Similarily, if $\mathbf{z}$ is a random vector:
  \[ p_{\mathbf{z}}(\mathbf{a}) \approx
  \frac{P(\textnormal{$z_i$ is in $[a_i, a_i + \nu]$ for all $i$})}{\nu^n}\]
  for very small values of $\nu$. Rigorous definitions are obtained by taking
  the one-sided limit $\nu \to 0^+$.
\end{frame}

\section{Joint and Marginal pdfs}

\begin{frame}{Joint and Marginal pdfs}
  The pdf $p(z_1, z_2, \ldots, z_n)$ of a random vector $\mathbf{z}$ is also
  called joint pdf, since it depends on all the components $z_i$.

  The marginal pdfs $p_{z_i}(z_i)$ are obtained by integrating over the other
  variables (that is $z_j$, with $j \ne i$). Example in two dimensions:
  \[ p_{z_1}(z_1) =  \int p_{\mathbf{z}}(z_1, z_2)dz_2.\]

  % Here we have used same notation for the random variable and its value. More
  % rigorous way would be to write:
  % \[ p_{z_1}(\nu_1) =  \int p(\nu_1, \nu_2)d\nu_2.\]
  % Earlier notation is often used in practice and we will also use it later.
\end{frame}

\section{Conditional Probabilities}

\begin{frame}{Conditional Probabilities}
  Conditional pdfs are obtained from any joint pdf by fixing the value of one (or
  several) of its components and multiplying the result with a normalization
  factor. Example: (conditional pdf of $z_2$ with $z_1$ given)
  \[
  p(z_2|z_1 = a) =
  \frac{ p_{\mathbf{z}}(a, z_2)}{\int p_{\mathbf{z}}(a, z_2)dz_2}.
  \]
  Using the definition of marginal pdf, this can be written as
  \[
  p(z_2|z_1 = a) =
  \frac{ p_{\mathbf{z}}(a, z_2)}{p_{z_1}(a)},
  \]
  which can be simplified further by omiting the subcsripts
  \[
  p(z_2|z_1 = a) =
  \frac{ p(a, z_2)}{p(a)}
  \]
\end{frame}

\begin{frame}{Conditional Probabilities}

  or by not introducing the quantity $a$ we can write it as
  \[
  p(z_2|z_1) =
  \frac{ p_{\mathbf{z}}(z_1, z_2)}{p(z_1)}.
  \]
  For the discrete case the integral is replaced by sum, that is
  \[
  P(z_2|z_1) = \frac{P_{\mathbf{z}}(z_1, z_2)}{P_{z_1}(z_1)}
  \]
  where
  \[
  P_{z_1}(z_1) = \sum_{z_2}P_{\mathbf{z}}(z_1, z_2).
  \]
\end{frame}

\section{Independence}

\begin{frame}{Independence}
  Two random variables $z_1$ and $z_2$ are said to be statistically independent
  if information about the value of one of them does not give any information
  about the value of the other. Formally this can be written as
  \[
  p(z_2|z_1) = p(z_2) \ \text{for every $z_1$ and $z_2$},
  \]
  since conditional probability assumes that the value of the other variable is
  fixed (and hence known). Substituting the formula of $p(z_1|z_2)$ we obtain
  two alternative ways to define independence:
  \[
  \frac{p(z_1, z_2)}{p(z_1)} = p(z_2)  \ \text{for every $z_1$ and $z_2$}
  \]
  or
  \[
  \ p(z_1, z_2) = p(z_1)p(z_2)  \ \text{for every $z_1$ and $z_2$}.
  \]
\end{frame}

%
% \section{Independence and covariance}
%
% \begin{frame}{Independence and covariance}
%   For independent variables $z_1$ and $z_2$ the relation
%   \[
%   E\{g_1(z_1)g_2(z_2)\} = E\{g_1(z_1)\}E\{g_2(z_2)\}
%   \]
%   holds for any functions $g_1$ and $g_2$
%   (rigorously speaking: for any measurable functions), which implies that
%   indepandent variables are uncorreleted (take $g_1(z) = g_2(z) = z$).
% \end{frame}

\section{Bayes' Rule}

\begin{frame}{Bayes' Rule}
  Bayes' rule gives us the probability distribution $p(\mathbf{s}|\mathbf{z})$
  when we know the distributions $p(\mathbf{z}|\mathbf{s})$ and $p(s)$. For
  continuously distributed random vector it is written as follows:
  \[
  p(\mathbf{s}|\mathbf{z}) =
  \frac{p(\mathbf{z}|\mathbf{s})p_{\mathbf{s}}(\mathbf{s})}{\int p(\mathbf{z}|\mathbf{s})p_{\mathbf{s}}(\mathbf{s})ds}.
  \]
  The distribution on the left is called \emph{posterior distribution} and the
  distribution $p_{\mathbf{s}}(\mathbf{s})$ on the right side is called
  \emph{prior distribution}.

  In the case of discretely distributed random vector the rule takes the form:
  \[
  P(\mathbf{s}|\mathbf{z}) =
  \frac{P(\mathbf{z}|\mathbf{s})P_{\mathbf{s}}(\mathbf{s})}{\sum_{\mathbf{s}} P(\mathbf{z}|\mathbf{s})P_{\mathbf{s}}(\mathbf{s})}.
  \]
\end{frame}

% \section{Non informative priors}
%
% \begin{frame}{Non informative priors}
%   definition
%   \[
%   p(\mathbf{s}|\mathbf{z}) =
%   \frac{p(\mathbf{z}|\mathbf{s})c}{\int p(\mathbf{z}|\mathbf{s})cds} =
%   \frac{p(\mathbf{z}|\mathbf{s})}{\int p(\mathbf{z}|\mathbf{s})ds}
%   \]
% \end{frame}
%
% \section{Bayesian inference as a incremental learning process}
%
% \begin{frame}{Bayesian inference as a incremental learning process}
%   Explanation
% \end{frame}
%
% %%%%%%%%%%%%%%%%%%%%%%%%%%%%%%%%%%%%%%%%%%%%%%%%%%%%%%%%%%%%%%%%%%%%%%%%%%%%
% \section{Template examples}
%
% \begin{frame}{Template examples}
%   Examples from the template on the next slides:
% \end{frame}
% %%%%%%%%%%%%%%%%%%%%%%%%%%%%%%%%%%%%%%%%%%%%%%%%%%%%%%%%%%%%%%%%%%%%%%%%%%%%
% %%%%%%%%% Some of the original default code below: %%%%%%%%%%%%%%%%%%%%%%%%%
% %%%%%%%%%%%%%%%%%%%%%%%%%%%%%%%%%%%%%%%%%%%%%%%%%%%%%%%%%%%%%%%%%%%%%%%%%%%%
%
% \section{Some \LaTeX{} Examples}
%
% \subsection{Tables and Figures}
%
% \begin{frame}{Tables and Figures}
%
% \begin{itemize}
% \item Use \texttt{tabular} for basic tables --- see Table~\ref{tab:widgets}, for example.
% \item You can upload a figure (JPEG, PNG or PDF) using the files menu.
% \item To include it in your document, use the \texttt{includegraphics} command (see the comment below in the source code).
% \end{itemize}
%
% % Commands to include a figure:
% %\begin{figure}
% %\includegraphics[width=\textwidth]{your-figure's-file-name}
% %\caption{\label{fig:your-figure}Caption goes here.}
% %\end{figure}
%
% \begin{table}
% \centering
% \begin{tabular}{l|r}
% Item & Quantity \\\hline
% Widgets & 42 \\
% Gadgets & 13
% \end{tabular}
% \caption{\label{tab:widgets}An example table.}
% \end{table}
%
% \end{frame}
%
% \subsection{Mathematics}
%
% \begin{frame}{Readable Mathematics}
%
% Let $X_1, X_2, \ldots, X_n$ be a sequence of independent and identically distributed random variables with $\text{E}[X_i] = \mu$ and $\text{Var}[X_i] = \sigma^2 < \infty$, and let
% $$S_n = \frac{X_1 + X_2 + \cdots + X_n}{n}
%       = \frac{1}{n}\sum_{i}^{n} X_i$$
% denote their mean. Then as $n$ approaches infinity, the random variables $\sqrt{n}(S_n - \mu)$ converge in distribution to a normal $\mathcal{N}(0, \sigma^2)$.
%
% \end{frame}

\section{Expectation}

\begin{frame}{Expectation}
  Expectation $E\{z\}$ of a random variable $z$ is a weighted average of the
  outcomes which the variable can obtain, with the weights been the individual
  propabilities (or values of the pdf in the continuous case).
  \[
  \text{Discrete case:} \qquad
  E\{z\} = \sum_{z} P_{z}(z)z
  \]
  \[
  \text{Continuous case:} \qquad
  E\{z\} = \int p_{z}(z)zdz
  \]
  The expectation can be also defined for random vectors componentwise:
  \[
  E\{\mathbf{z}\} =
  \begin{pmatrix} E\{z_1\} \\ E\{z_2\} \\ \vdots \\ E\{z_n\} \end{pmatrix}
  \]
  where $E\{z_i\}$ is defined as above, with $z$ replaced by $z_i$ for all $i$.
  % (write the components?)\\
  % it is linear:
  % \[
  % E\{a\mathbf{z} + b\mathbf{z}\}  = aE\{\mathbf{z}\} + bE\{\mathbf{z}\}.
  % \]
  % Similarily, for any matrix $\mathbf{M}$
  % \[
  % E\{\mathbf{Mz}\} = \mathbf{M}E\{\mathbf{z}\}.
  % \]
\end{frame}

\section{Variance and Covariance}

\begin{frame}{Variance and Covariance}
  Variance in some sense tells how close the probability mass is concentrated
  near the expected value. It is defined as follows:
  \[
  var(z)= E\{z^2\}-(E\{z\})^2.
  \]
  Alternatively it can be written as
  \[
   var(z) = E\{(z-E\{z\})^2\}.
  \]
  When we have two variables we can define their covariance:
  \[
  cov(z_1, z_2) = E\{z_1 z_2\} - E\{z_1\}E\{z_2\}
  \]
  which measures how well we can predict the value of one variable from another
  by using simple linear predictor.

  % \section{Correlation Coefficient}
  Covariance is often normalized and this quantity is the correlation
  coefficient:
  \[
  corr(z_1,z_2) = \frac{cov(z_1, z_2)}{\sqrt{var(z_1)var(z_2)}}.
  \]
\end{frame}

% \section{Covariance Matrix}
%
% \begin{frame}{Covariance Matrix}
%   \[
%   \mathbf{C}(\mathbf{z}) =
%   \begin{pmatrix}
%     cov(z_1, z_1) \ cov(z_1, z_2) \ldots cov(z_1, z_n) \\
%     cov(z_2, z_1) \ cov(z_2, z_2) \ldots cov(z_2, z_n) \\
%     \vdots \qquad \qquad \ddots \quad \vdots \\
%     cov(z_n, z_1) \ cov(z_n, z_2) \ldots cov(z_n, z_n)
%   \end{pmatrix}
%   \]
%   or more concisely:
%   \[ \mathbf{C}(\mathbf{z}) =
%   E\{\mathbf{zz}^T\} - E\{\mathbf{z}\} E\{\mathbf{z}\}^T\]
% \end{frame}

\section{Likelihood and Log-likelihood}

\begin{frame}{Likelihood and Log-likelihood}
  Given $n$ samples $z(1), \ldots , z(n)$ of a random variable variable $z$, which
  is assumed to have probability distribution $p(z, \alpha)$ for some unknown
  parameter $\alpha$, we can define the likelihood:
  \[
  p(z(1), \ldots , z(n)|\alpha) = p(z(1)|\alpha)\times \ldots \times p(z(n)|\alpha)
  \]
  and the log-likelihood:
  \[
  \log p(z(1), \ldots , z(n)|\alpha) = \log p(z(1)|\alpha) + \ldots + \log p(z(n)|\alpha).
  \]
  These are functions of the parameter $\alpha$, when the model $p(z, \alpha)$
  and the samples $z(1), \ldots , z(n)$ are fixed.
\end{frame}

\section{Maximum Likelihood and Maximum a Posteriori}

\begin{frame}{Maximum Likelihood and Maximum a Posteriori}
  In maximum likelihood and maximum a posteriori estimates one tries to find
  the value of the parameter $\alpha$ which is in some sense optimal for
  observing the samples $z(1), \ldots , z(n)$.

  Using the Bayes' rule we see that
  \[
  p(\alpha| z(1), \ldots , z(n)) = \frac{p(z(1), \ldots , z(n)|\alpha)p(\alpha)}{p(z(1), \ldots , z(n))}.
  \]
  The maximum a posteriori estimate $\hat{\alpha}$ is the one that maximizes
  the value of the posterior probability for the given prior $p(\alpha)$.

  Maximum likelihood estimate $\hat{\alpha}$ on the other hand is the one which
  maximizes the likelihood.

  If we choose a flat prior, that is $p(\alpha) =$ constant, these estimates are
  the same.
\end{frame}

\section{References}

\begin{frame}
\frametitle{References}
\footnotesize{
\begin{thebibliography}{09}
\bibitem[1]{p1} Aapo Hyvärinen, Jarmo Hurri, and Patrik O. Hoyer
\newblock Natural Image Statistics: A Probabilistic Approach to Early Computational Vision
\newblock \emph{Springer-Verlag}, 2009
\end{thebibliography}
}
\end{frame}

\end{document}
