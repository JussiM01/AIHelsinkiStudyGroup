\documentclass[10pt,a4paper]{amsart}
\usepackage[latin1]{inputenc}
\usepackage[T1]{fontenc}
\usepackage[finnish]{babel}
\usepackage{amsfonts}
\usepackage{amsmath}
\usepackage{amssymb}
\usepackage{calc}
\def\viiva #1 #2 {\mathop{\Big/}\limits_{\!\!\!{#1}}^{\>\,{#2}}}



\newcommand{\R}{\mathbb{R}}
\newcommand{\C}{\mathbb{C}}
\newcommand{\Q}{\mathbb{Q}}
\newcommand{\N}{\mathbb{N}}
\newcommand{\Z}{\mathbb{Z}}


\setlength{\parindent}{0pt}
\newenvironment{tehtratk}%
             {\begin{list}{\arabic{enumi}.}{\usecounter{enumi}%
              \setlength{\labelsep}{0.5em}%
              \settowidth{\labelwidth}{\arabic{enumi}.}%
              \setlength{\leftmargin}{\labelwidth+\labelsep}}}%
             {\end{list}}
\begin{document}

\textbf{Natural Image Statistics\\
Homework for chapter 4\\
Model solutions (Jussi Martin)}

\vspace{1cm}

\begin{tehtratk}
\item[\textbf{1.}] (Math exercise 1 from chapter 4).\\
\\
\textbf{Solution:}\\
If we right the conditional probabilty as
\[ p(z_2|z_1 = a) =
\frac{p_{\mathbf{z}}(a, z_2)}{\int p_{\mathbf{z}}(a, z_2)dz_2}\]
we see that integration over $z_1$ gives us
\[ \int p(z_2|z_1 = a)dz_2 =
\int \frac{p_{\mathbf{z}}(a, z_2)}{\int p_{\mathbf{z}}(a, z_2)dz_2}dz_2 =
\frac{\int p_{\mathbf{z}}(a, z_2)dz_2}{\int p_{\mathbf{z}}(a, z_2)dz_2} = 1\]
since the denominator is just a constant.
\vspace{0.3cm}

\item[\textbf{2.}] (Math exercise 2 from chapter 4).\\
\\
\textbf{Solution:}\\
\\

\vspace{0.3cm}
\item[\textbf{3.}] (Math exercise 6 from chapter 4).\\
\\
\textbf{Solution:}\\
\\

\end{tehtratk}
\end{document}
