\documentclass[10pt,a4paper]{amsart}
\usepackage[latin1]{inputenc}
\usepackage[T1]{fontenc}
\usepackage[finnish]{babel}
\usepackage{amsfonts}
\usepackage{amsmath}
\usepackage{amssymb}
\usepackage{calc}
\def\viiva #1 #2 {\mathop{\Big/}\limits_{\!\!\!{#1}}^{\>\,{#2}}}



\newcommand{\R}{\mathbb{R}}
\newcommand{\C}{\mathbb{C}}
\newcommand{\Q}{\mathbb{Q}}
\newcommand{\N}{\mathbb{N}}
\newcommand{\Z}{\mathbb{Z}}


\setlength{\parindent}{0pt}
\newenvironment{tehtratk}%
             {\begin{list}{\arabic{enumi}.}{\usecounter{enumi}%
              \setlength{\labelsep}{0.5em}%
              \settowidth{\labelwidth}{\arabic{enumi}.}%
              \setlength{\leftmargin}{\labelwidth+\labelsep}}}%
             {\end{list}}
\begin{document}

\textbf{Natural Image Statistics\\
Homework for chapter 4\\
Model solutions (Jussi Martin)}

\vspace{1cm}

\begin{tehtratk}
\item[\textbf{1.}] Math exercise 1 from chapter 4.\\
\\
\textbf{Solution:} If we right the conditional probability as
\[ p(z_2|z_1 = a) =
\frac{p_{\mathbf{z}}(a, z_2)}{\int p_{\mathbf{z}}(a, z_2)dz_2}\]
we see that integration over $z_2$ gives us
\[ \int p(z_2|z_1 = a)dz_2 =
\int \frac{p_{\mathbf{z}}(a, z_2)}{\int p_{\mathbf{z}}(a, z_2)dz_2}dz_2 =
\frac{\int p_{\mathbf{z}}(a, z_2)dz_2}{\int p_{\mathbf{z}}(a, z_2)dz_2} = 1\]
since the denominator is just a constant.
\vspace{0.3cm}

\item[\textbf{2.}] Math exercise 2 from chapter 4.\\
\\
\textbf{Solution:} Lets denote the random variable by $z$ (in the sequel I will
denote also it's values by $z$). Since it is distributed uniformly on the
interval $[a, b]$ we have
\[
p(z) = \frac{1}{b - a} \quad \text{for every $z$ in $[a, b]$}.
\]
It's mean is
\[
E\{z\} = \int_a^b p(z)zdz = \frac{1}{b - a}\int_a^b zdz =
\frac{1}{b - a}\Big(\frac{b^2}{2} - \frac{b^2}{2}\Big).
\]
Since
\[
b^2 - a^2 = (b + a)(b - a)
\]
we get
\[
E\{z\} =
\frac{1}{2}\frac{(b + a)(b - a)}{b - a} = \frac{b + a}{2}.
\]
It's variance is
\[
\text{var}(z) = E\{z^2\} - (E\{z\})^2 = \int_a^b p(z)z^2dz  - (E\{z\})^2.
\]
Using the previous result for the mean we get
\[
\text{var}(z) = \int_a^b p(z)z^2dz  - \frac{(b + a)^2}{4} =
\frac{1}{b - a}\int_a^b z^2dz - \frac{(b + a)^2}{4}.
\]
Solving the integral we get
\[
\text{var}(z) =
\frac{1}{b - a}\Big(\frac{b^3}{3} - \frac{a^3}{3}\Big)- \frac{(b + a)^2}{4}
\]
and since
\[
b^3 - a^3 = (b -a)(b^2 + ab + a^2)
\]
we see that
\[
\text{var}(z) = \frac{(b^2 + ab + a^2)}{3} - \frac{(b + a)^2}{4}
= \frac{4b^2 + 4ab + 4a^2}{12} -\frac{3b^2 + 6ab + 3b^2}{12}.\]
Which reducies to:
\[
\text{var}(z) = \frac{b^2 - 2ab - a^2}{12} = \frac{(b - a)}{12}^2.
\]
\vspace{0.3cm}
\item[\textbf{3.}] Math exercise 6 from chapter 4.\\
\\
\textbf{Solution:} Now the likelihood is
\[
p(z|\alpha) = \frac{1}{\sqrt{2 \pi}}\exp(-\frac{1}{2}(z - \alpha)^2)
\]
and the log-likelihood is
\[
\log p(z|\alpha) = -\frac{1}{2}(z - \alpha)^2 - \text{const.}
\]
Since the logarithm is incresing function, the value which maximizes the
log-likelihood also maximizes the likelihood. Furthermore, the log-likelihood
function is a down opening parabola and hence the maximum is achieved in the
uniq point on top of the parabola. This point is also a point where derivative
of function
\[f(\alpha) = -\frac{1}{2}(z - \alpha)^2\]
vanishes.
The maximum value of the likelihood is thus found by seting $f'(\alpha) = 0$,
which yields:
\[
-z + \alpha = 0 \quad \Leftrightarrow \quad \alpha = z.
\]
\end{tehtratk}
\end{document}
